\subsection{AutomationRequestValidatorService}\label{automationrequestvalidatorservice}

This service provides a simple sanity check on external automation
requests. It also queues requests and tags them with unique identifiers,
feeding them into the system one at a time.

This service has two states: \textbf{idle} and \textbf{busy}. In both
states, when a non \emph{AutomationRequest} message is received, a local
memory store is updated to maintain a list of all available tasks,
vehicle configurations, vehicle states, zones, and operating regions.

Upon reception of an \emph{AutomationRequest} message, this service
ensures that such a request can be carried out by checking its local
memory store for existence of the requested vehicles, tasks, and
operating region. If the request includes vehicles, tasks, or an
operating region that has not previously been defined, this service will
publish an error message.

Upon determination that the \emph{AutomationRequest} includes only
vehicles, tasks, and an operating region that have previously been
defined, this service creates a \emph{UniqueAutomationRequest} with a
previously unused unique identifier. If in the \textbf{idle} state, this
service will immediately publish the \emph{UniqueAutomationRequest}
message and transition to the \textbf{busy} state. If already in the
\textbf{busy} state, the \emph{UniqueAutomationRequest} will be added to
the end of a queue.

When this service receives either an error message (indicating that the
\emph{UniqueAutomationRequest} cannot be fulfilled or a corresponding
\emph{UniqueAutomationResponse}), it will publish the same message. If
in the \textbf{idle} state, it will remain in the \textbf{idle} state.
If in the \textbf{busy} state, it will remove from the queue the request
that was just fulfilled and then send the next
\emph{UniqueAutomationRequest} in the queue. If the queue is empty, this
service transitions back to the \textbf{idle} state.

This service also includes a parameter that allows an optional
\emph{timeout} value to be set. When a \emph{UniqueAutomationRequest} is
published, a timer begins. If the \emph{timeout} has been reached before
a \emph{UniqueAutomationResponse} is received, an error is assumed to
have occured and this service removes the pending
\emph{UniqueAutomationRequest} from the queue and attempts to send the
next in the queue or transition to \textbf{idle} if the queue is empty.

\begin{longtable}[c]{@{}ll@{}}
\caption{Table of messages that the
\emph{AutomationRequestValidatorService} receives and
processes.}\tabularnewline
\toprule
\begin{minipage}[b]{0.29\columnwidth}\raggedright\strut
Message Subscription
\strut\end{minipage} &
\begin{minipage}[b]{0.65\columnwidth}\raggedright\strut
Description
\strut\end{minipage}\tabularnewline
\midrule
\endfirsthead
\toprule
\begin{minipage}[b]{0.29\columnwidth}\raggedright\strut
Message Subscription
\strut\end{minipage} &
\begin{minipage}[b]{0.65\columnwidth}\raggedright\strut
Description
\strut\end{minipage}\tabularnewline
\midrule
\endhead
\begin{minipage}[t]{0.29\columnwidth}\raggedright\strut
\emph{AutomationRequest}
\strut\end{minipage} &
\begin{minipage}[t]{0.65\columnwidth}\raggedright\strut
Primary message to request a set of Tasks to be completed by a set of
vehicles in a particular airspace configuration (described by an
\emph{OperatingRegion}).
\strut\end{minipage}\tabularnewline
\begin{minipage}[t]{0.29\columnwidth}\raggedright\strut
\begin{verbatim}
(2 ms work)
\end{verbatim}
\strut\end{minipage} &
\begin{minipage}[t]{0.65\columnwidth}\raggedright\strut
Determines validity of \emph{AutomationRequest}. \textbf{idle}
-\textgreater{} \textbf{busy}, emit \emph{UniqueAutomationRequest}
\textbf{busy} -\textgreater{} \textbf{busy}, add
\emph{UniqueAutomationRequest} to queue
\strut\end{minipage}\tabularnewline
\begin{minipage}[t]{0.29\columnwidth}\raggedright\strut
\emph{EntityConfiguration}
\strut\end{minipage} &
\begin{minipage}[t]{0.65\columnwidth}\raggedright\strut
Vehicle capabilities (e.g.~allowable speeds) are described by entity
configuration messages. Any vehicle requested in an
\emph{AutomationRequest} must previously be described by an associated
\emph{EntityConfiguration}.
\strut\end{minipage}\tabularnewline
\begin{minipage}[t]{0.29\columnwidth}\raggedright\strut
\begin{verbatim}
(0 ms work)
\end{verbatim}
\strut\end{minipage} &
\begin{minipage}[t]{0.65\columnwidth}\raggedright\strut
Add to internal storage for use in validation step. \textbf{idle}
-\textgreater{} \textbf{idle}, add to storage \textbf{busy}
-\textgreater{} \textbf{busy}, add to storage
\strut\end{minipage}\tabularnewline
\begin{minipage}[t]{0.29\columnwidth}\raggedright\strut
\emph{EntityState}
\strut\end{minipage} &
\begin{minipage}[t]{0.65\columnwidth}\raggedright\strut
Describes the actual state of a vehicle in the system including
position, speed, and fuel status. Each vehicle in an
\emph{AutomationRequest} must have reported its state.
\strut\end{minipage}\tabularnewline
\begin{minipage}[t]{0.29\columnwidth}\raggedright\strut
\begin{verbatim}
(0 ms work)
\end{verbatim}
\strut\end{minipage} &
\begin{minipage}[t]{0.65\columnwidth}\raggedright\strut
Add to internal storage for use in validation step. \textbf{idle}
-\textgreater{} \textbf{idle}, add to storage \textbf{busy}
-\textgreater{} \textbf{busy}, add to storage
\strut\end{minipage}\tabularnewline
\begin{minipage}[t]{0.29\columnwidth}\raggedright\strut
\emph{Task}
\strut\end{minipage} &
\begin{minipage}[t]{0.65\columnwidth}\raggedright\strut
Details a particular task that will be referenced (by ID) in an
\emph{AutomationRequest}.
\strut\end{minipage}\tabularnewline
\begin{minipage}[t]{0.29\columnwidth}\raggedright\strut
\begin{verbatim}
(0 ms work)
\end{verbatim}
\strut\end{minipage} &
\begin{minipage}[t]{0.65\columnwidth}\raggedright\strut
Add to internal storage for use in validation step. \textbf{idle}
-\textgreater{} \textbf{idle}, add to storage \textbf{busy}
-\textgreater{} \textbf{busy}, add to storage
\strut\end{minipage}\tabularnewline
\begin{minipage}[t]{0.29\columnwidth}\raggedright\strut
\emph{TaskInitialized}
\strut\end{minipage} &
\begin{minipage}[t]{0.65\columnwidth}\raggedright\strut
Indicates that a particular task is ready to proceed with the task
assignment sequence. Each task requested in the \emph{AutomationRequest}
must be initialized before a \emph{UniqueAutomationRequest} is
published.
\strut\end{minipage}\tabularnewline
\begin{minipage}[t]{0.29\columnwidth}\raggedright\strut
\begin{verbatim}
(0 ms work)
\end{verbatim}
\strut\end{minipage} &
\begin{minipage}[t]{0.65\columnwidth}\raggedright\strut
Add to internal storage for use in validation step \textbf{idle}
-\textgreater{} \textbf{idle}, add to storage \textbf{busy}
-\textgreater{} \textbf{busy}, add to storage
\strut\end{minipage}\tabularnewline
\begin{minipage}[t]{0.29\columnwidth}\raggedright\strut
\emph{KeepOutZone}
\strut\end{minipage} &
\begin{minipage}[t]{0.65\columnwidth}\raggedright\strut
Polygon description of a region in which vehicles must not travel. If
referenced by the \emph{OperatingRegion} in the
\emph{AutomationRequest}, zone must exist for request to be valid.
\strut\end{minipage}\tabularnewline
\begin{minipage}[t]{0.29\columnwidth}\raggedright\strut
\begin{verbatim}
(0 ms work)
\end{verbatim}
\strut\end{minipage} &
\begin{minipage}[t]{0.65\columnwidth}\raggedright\strut
Add to internal storage for use in validation step. \textbf{idle}
-\textgreater{} \textbf{idle}, add to storage \textbf{busy}
-\textgreater{} \textbf{busy}, add to storage
\strut\end{minipage}\tabularnewline
\begin{minipage}[t]{0.29\columnwidth}\raggedright\strut
\emph{KeepInZone}
\strut\end{minipage} &
\begin{minipage}[t]{0.65\columnwidth}\raggedright\strut
Polygon description of a region in which vehicles must remain during
travel. If referenced by the \emph{OperatingRegion} in the
\emph{AutomationRequest}, zone must exist for request to be valid.
\strut\end{minipage}\tabularnewline
\begin{minipage}[t]{0.29\columnwidth}\raggedright\strut
\begin{verbatim}
(0 ms work)
\end{verbatim}
\strut\end{minipage} &
\begin{minipage}[t]{0.65\columnwidth}\raggedright\strut
Add to internal storage for use in validation step. \textbf{idle}
-\textgreater{} \textbf{idle}, add to storage \textbf{busy}
-\textgreater{} \textbf{busy}, add to storage
\strut\end{minipage}\tabularnewline
\begin{minipage}[t]{0.29\columnwidth}\raggedright\strut
\emph{OperatingRegion}
\strut\end{minipage} &
\begin{minipage}[t]{0.65\columnwidth}\raggedright\strut
Collection of \emph{KeepIn} and \emph{KeepOut} zones that describe the
allowable space for vehicular travel. Must be defined for
\emph{AutomationRequest} to be valid.
\strut\end{minipage}\tabularnewline
\begin{minipage}[t]{0.29\columnwidth}\raggedright\strut
\begin{verbatim}
(0 ms work)
\end{verbatim}
\strut\end{minipage} &
\begin{minipage}[t]{0.65\columnwidth}\raggedright\strut
Add to internal storage for use in validation step. \textbf{idle}
-\textgreater{} \textbf{idle}, add to storage \textbf{busy}
-\textgreater{} \textbf{busy}, add to storage
\strut\end{minipage}\tabularnewline
\begin{minipage}[t]{0.29\columnwidth}\raggedright\strut
\emph{UniqueAutomationResponse}
\strut\end{minipage} &
\begin{minipage}[t]{0.65\columnwidth}\raggedright\strut
Completed response from the rest of the task assignment process.
Indicates that the next \emph{AutomationRequest} is ready to be
processed.
\strut\end{minipage}\tabularnewline
\begin{minipage}[t]{0.29\columnwidth}\raggedright\strut
\begin{verbatim}
(1 ms work)
\end{verbatim}
\strut\end{minipage} &
\begin{minipage}[t]{0.65\columnwidth}\raggedright\strut
If response ID does not match request ID at top of queue, ignore and
remain in current state. Otherwise: \textbf{idle} -\textgreater{}
\textbf{idle}, normal operation should preclude receiving this message
in the \textbf{idle} state \textbf{busy}, emit corresponding
\emph{AutomationResponse} If queue is empty, \textbf{busy}
-\textgreater{} \textbf{idle}, else \textbf{busy} -\textgreater{}
\textbf{busy}, emit request message at top of queue
\strut\end{minipage}\tabularnewline
\bottomrule
\end{longtable}

\begin{longtable}[c]{@{}ll@{}}
\caption{Table of messages that the
\emph{AutomationRequestValidatorService} publishes.}\tabularnewline
\toprule
\begin{minipage}[b]{0.29\columnwidth}\raggedright\strut
Message Publication
\strut\end{minipage} &
\begin{minipage}[b]{0.65\columnwidth}\raggedright\strut
Description
\strut\end{minipage}\tabularnewline
\midrule
\endfirsthead
\toprule
\begin{minipage}[b]{0.29\columnwidth}\raggedright\strut
Message Publication
\strut\end{minipage} &
\begin{minipage}[b]{0.65\columnwidth}\raggedright\strut
Description
\strut\end{minipage}\tabularnewline
\midrule
\endhead
\begin{minipage}[t]{0.29\columnwidth}\raggedright\strut
\emph{UniqueAutomationRequest}
\strut\end{minipage} &
\begin{minipage}[t]{0.65\columnwidth}\raggedright\strut
A duplicate message to an external \emph{AutomationRequest} but only
published if the request is determined to be valid. Also includes a
unique identifier to match to the corresponding response.
\strut\end{minipage}\tabularnewline
\begin{minipage}[t]{0.29\columnwidth}\raggedright\strut
\emph{ServiceStatus}
\strut\end{minipage} &
\begin{minipage}[t]{0.65\columnwidth}\raggedright\strut
Error message when a request is determined to be invalid. Includes human
readable error message that highlights which portion of the
\emph{AutomationRequest} was invalid.
\strut\end{minipage}\tabularnewline
\begin{minipage}[t]{0.29\columnwidth}\raggedright\strut
\emph{AutomationResponse}
\strut\end{minipage} &
\begin{minipage}[t]{0.65\columnwidth}\raggedright\strut
Upon reception of a completed \emph{UniqueAutomationResponse}, this
message is published as a response to the original request.
\strut\end{minipage}\tabularnewline
\bottomrule
\end{longtable}

\subsection{TaskManagerService}\label{taskmanagerservice}

The \emph{TaskManagerService} is a very straight-forward service. Upon
reception of a Task message, it will send the appropriate
\emph{CreateNewService} message. To do so, it catalogues all entity
configurations and current states; areas, lines, and points of interest;
and current waypoint paths for each vehicle. This information is stored
in local memory and appended as part of the \emph{CreateNewService}
message which allows new Tasks to immediately be informed of all
relevant information needed to carry out a Task.

When \emph{TaskManagerService} receives a \emph{RemoveTasks} message, it
will form the appropriate \emph{KillService} message to properly destroy
the service that was created to fulfill the original Task.

\begin{longtable}[c]{@{}ll@{}}
\caption{Table of messages that the \emph{TaskManagerService} receives
and processes.}\tabularnewline
\toprule
\begin{minipage}[b]{0.29\columnwidth}\raggedright\strut
Message Subscription
\strut\end{minipage} &
\begin{minipage}[b]{0.65\columnwidth}\raggedright\strut
Description
\strut\end{minipage}\tabularnewline
\midrule
\endfirsthead
\toprule
\begin{minipage}[b]{0.29\columnwidth}\raggedright\strut
Message Subscription
\strut\end{minipage} &
\begin{minipage}[b]{0.65\columnwidth}\raggedright\strut
Description
\strut\end{minipage}\tabularnewline
\midrule
\endhead
\begin{minipage}[t]{0.29\columnwidth}\raggedright\strut
\emph{Task}
\strut\end{minipage} &
\begin{minipage}[t]{0.65\columnwidth}\raggedright\strut
Primary message that describes a particular task. The task manager will
make the appropriate service creation message to build a service that
directly handles this requested Task.
\strut\end{minipage}\tabularnewline
\begin{minipage}[t]{0.29\columnwidth}\raggedright\strut
\begin{verbatim}
(1 ms work)
\end{verbatim}
\strut\end{minipage} &
\begin{minipage}[t]{0.65\columnwidth}\raggedright\strut
Emit \emph{CreateNewService} message
\strut\end{minipage}\tabularnewline
\begin{minipage}[t]{0.29\columnwidth}\raggedright\strut
\emph{RemoveTasks}
\strut\end{minipage} &
\begin{minipage}[t]{0.65\columnwidth}\raggedright\strut
Indicates that Task is no longer needed and will not be included in
future \emph{AutomationRequest} messages. Task manager will send the
proper \emph{KillService} message to remove the service that was
constructed to handle the requested Task.
\strut\end{minipage}\tabularnewline
\begin{minipage}[t]{0.29\columnwidth}\raggedright\strut
\begin{verbatim}
(1 ms work)
\end{verbatim}
\strut\end{minipage} &
\begin{minipage}[t]{0.65\columnwidth}\raggedright\strut
Emit \emph{KillService} message corresponding to \emph{Task}
\strut\end{minipage}\tabularnewline
\begin{minipage}[t]{0.29\columnwidth}\raggedright\strut
\emph{EntityConfiguration}
\strut\end{minipage} &
\begin{minipage}[t]{0.65\columnwidth}\raggedright\strut
Vehicle capabilities (e.g.~allowable speeds) are described by entity
configuration messages. New Tasks are informed of all known entities
upon creation.
\strut\end{minipage}\tabularnewline
\begin{minipage}[t]{0.29\columnwidth}\raggedright\strut
\begin{verbatim}
(0 ms work)
\end{verbatim}
\strut\end{minipage} &
\begin{minipage}[t]{0.65\columnwidth}\raggedright\strut
Store to report during \emph{Task} creation
\strut\end{minipage}\tabularnewline
\begin{minipage}[t]{0.29\columnwidth}\raggedright\strut
\emph{EntityState}
\strut\end{minipage} &
\begin{minipage}[t]{0.65\columnwidth}\raggedright\strut
Describes the actual state of a vehicle in the system including
position, speed, and fuel status. New Tasks are informed of all known
entity states upon creation.
\strut\end{minipage}\tabularnewline
\begin{minipage}[t]{0.29\columnwidth}\raggedright\strut
\begin{verbatim}
(0 ms work)
\end{verbatim}
\strut\end{minipage} &
\begin{minipage}[t]{0.65\columnwidth}\raggedright\strut
Store to report during \emph{Task} creation
\strut\end{minipage}\tabularnewline
\begin{minipage}[t]{0.29\columnwidth}\raggedright\strut
\emph{AreaOfInterest} \emph{LineOfInterest} \emph{PointOfInterest}
\strut\end{minipage} &
\begin{minipage}[t]{0.65\columnwidth}\raggedright\strut
Describes known geometries of areas, lines, and points. New Tasks are
informed of all such named areas upon creation.
\strut\end{minipage}\tabularnewline
\begin{minipage}[t]{0.29\columnwidth}\raggedright\strut
\begin{verbatim}
(0 ms work)
\end{verbatim}
\strut\end{minipage} &
\begin{minipage}[t]{0.65\columnwidth}\raggedright\strut
Store to report during \emph{Task} creation
\strut\end{minipage}\tabularnewline
\begin{minipage}[t]{0.29\columnwidth}\raggedright\strut
\emph{MissionCommand}
\strut\end{minipage} &
\begin{minipage}[t]{0.65\columnwidth}\raggedright\strut
Describes current set of waypoints that a vehicle is following. New
Tasks are informed of all known current waypoint routes upon creation.
\strut\end{minipage}\tabularnewline
\begin{minipage}[t]{0.29\columnwidth}\raggedright\strut
\begin{verbatim}
(0 ms work)
\end{verbatim}
\strut\end{minipage} &
\begin{minipage}[t]{0.65\columnwidth}\raggedright\strut
Store to report during \emph{Task} creation
\strut\end{minipage}\tabularnewline
\bottomrule
\end{longtable}

\begin{longtable}[c]{@{}ll@{}}
\caption{Table of messages that the \emph{TaskManagerService}
publishes.}\tabularnewline
\toprule
\begin{minipage}[b]{0.29\columnwidth}\raggedright\strut
Message Publication
\strut\end{minipage} &
\begin{minipage}[b]{0.65\columnwidth}\raggedright\strut
Description
\strut\end{minipage}\tabularnewline
\midrule
\endfirsthead
\toprule
\begin{minipage}[b]{0.29\columnwidth}\raggedright\strut
Message Publication
\strut\end{minipage} &
\begin{minipage}[b]{0.65\columnwidth}\raggedright\strut
Description
\strut\end{minipage}\tabularnewline
\midrule
\endhead
\begin{minipage}[t]{0.29\columnwidth}\raggedright\strut
\emph{CreateNewService}
\strut\end{minipage} &
\begin{minipage}[t]{0.65\columnwidth}\raggedright\strut
Primary message published by the Task Manager to dynamically build a new
Task from an outside description of such a Task.
\strut\end{minipage}\tabularnewline
\begin{minipage}[t]{0.29\columnwidth}\raggedright\strut
\emph{KillService}
\strut\end{minipage} &
\begin{minipage}[t]{0.65\columnwidth}\raggedright\strut
When Tasks are no longer needed, the Task Manager will correctly clean
up and destroy the service that was built to handle the original Task.
\strut\end{minipage}\tabularnewline
\bottomrule
\end{longtable}

\subsection{Task}\label{task}

A \emph{Task} forms the core functionality of vehicle behavior. It is
the point at which a vehicle (or set of vehicles) is dedicated to a
singular goal. During \emph{Task} execution, a wide spectrum of behavior
is allowed, including updating waypoints and steering sensors. As part
of the core services, this general \emph{Task} description stands in for
all \emph{Tasks} running in the system.

The general \emph{Task} interaction with the rest of the task assignment
pipeline is complex. It is the aggregation of each \emph{Task's}
possibilities that defines the complexity of the overall mission
assignment. These \emph{Task} possibilities are called \emph{options}
and they describe the precise ways that a \emph{Task} could unfold. For
example, a \emph{LineSearchTask} could present two options to the
system: 1) search the line from East-to-West and 2) search the line from
West-to-East. Either is valid and a selection of one of these options
that optimizes overall mission efficiency is the role of the assignment
service.

A general \emph{Task} is comprised of up to nine states with each state
corresponding to a place in the message sequence that carries out the
task assignment pipeline. The states for a \emph{Task} are:

\begin{itemize}
\tightlist
\item
  \textbf{Init}: This is the state that all \emph{Tasks} start in and
  remain until all internal initialization is complete. For example, a
  \emph{Task} may need to load complex terrain or weather data upon
  creation and will require some (possibly significant) start-up time.
  When a \emph{Task} has completed its internal initialization, it must
  report transition from this state via the \emph{TaskInitialized}
  message.
\item
  \textbf{Idle}: This represents the state of a \emph{Task} after
  initialization, but before any requests have been made that include
  the \emph{Task}. \emph{UniqueAutomationRequest} messages trigger a
  transition from this state into the \textbf{SensorRequest} state.
\item
  \textbf{SensorRequest}: When a \emph{Task} is notified of its
  inclusion (by noting the presence of its ID in the \emph{Tasks} list
  of an \emph{UniqueAutomationRequest} message), it can request
  calculations that pertain to the sensors onboard the vehicles that are
  also included in the \emph{UniqueAutomationRequest} message. While
  waiting for a response from the \emph{SensorManagerService}, a
  \emph{Task} is in the \textbf{SensorRequest} state and will remain so
  until the response from the \emph{SensorManagerService} is received.
\item
  \textbf{OptionRoutes}: After the \emph{SensorManagerService} has
  replied with the appropriate sensor calculations, the \emph{Task} can
  request waypoints from the \emph{RouteAggregatorService} that carry
  out the on-\emph{Task} goals. For example, an \emph{AreaSearchTask}
  can request routes from key surveillance positions that ensure sensor
  coverage of the entire area. The \emph{Task} remains in the
  \textbf{OptionRoutes} state until the \emph{RouteAggregatorService}
  replies.
\item
  \textbf{OptionsPublished}: When routes are returned to the
  \emph{Task}, it will utilize all route and sensor information to
  identify and publish the applicable \emph{TaskOptions}. The
  determination of \emph{TaskOptions} is key to overall mission
  performance and vehicle behavior. It is from this list of options that
  the assignment will select in order to perform this particular
  \emph{Task}. After publication of the options, a \emph{Task} waits in
  the \textbf{OptionsPublished} state until the
  \emph{TaskImplementationRequest} message is received, whereupon it
  switches to \textbf{FinalRoutes}.
\item
  \textbf{FinalRoutes}: Upon reception of a
  \emph{TaskImplementationRequest}, a \emph{Task} is informed of the
  option that was selected by the assignment service. At this point, a
  \emph{Task} must create the final set of waypoints that include both
  \emph{enroute} and \emph{on-task} waypoints from the specified vehicle
  location. The \emph{Task} is required to create the \emph{enroute}
  waypoints since a route refinement is possible, taking advantage of
  the concrete prior position of the selected vehicle. The \emph{Task}
  remains in the \textbf{FinalRoutes} state until the route request is
  fulfilled by the \emph{RouteAggregatorService}.
\item
  \textbf{OptionSelected}: When the final waypoints are returned from
  the \emph{RouteAggregatorService}, the \emph{Task} publishes a
  complete \emph{TaskImplementationResponse} message. A \emph{Task} will
  remain in this state until an \emph{EntityState} message includes this
  \emph{Task} ID in its \emph{AssociatedTaskList}. If during this state,
  a subsequent \emph{UniqueAutomationRequest} is made, the \emph{Task}
  returns to the \textbf{SensorRequest} state and immediately attempts
  to fulfill the requirements of the new \emph{UniqueAutomationRequest}.
  This behavior implies that a \emph{Task} can only be part of a single
  \emph{AutomationRequest} and subsequent requests always override
  previous requests.
\item
  \textbf{Active}: If the \emph{Task} is in the \textbf{OptionSelected}
  state and an \emph{EntityState} message is received which includes the
  \emph{Task} ID in the \emph{AssociatedTaskList}, then the \emph{Task}
  switches to the \textbf{Active} state and is allowed to publish new
  waypoints and sensor commands at will. A \emph{Task} remains in the
  \textbf{Active} state until a subsequent \emph{EntityState} message
  does \emph{not} list the \emph{Task} ID in its
  \emph{AssociatedTaskList}. At which point, a transition to
  \textbf{Completed} is made. Note that a \emph{Task} can reliquish
  control indirectly by sending the vehicle to a waypoint not tagged
  with its own ID. Likewise, it can maintain control indefinitely by
  ensuring that the vehicle only ever go to a waypoint that includes its
  ID. If a \emph{UniqueAutomationRequest} message that includes this
  \emph{Task} ID is received in the \textbf{Active} state, it
  transitions to the \textbf{Completed} state.
\item
  \textbf{Completed}: In this state, the \emph{Task} publishes a
  \emph{TaskComplete} message and then immediately transitions to the
  \textbf{Idle} state.
\end{itemize}

\begin{longtable}[c]{@{}ll@{}}
\caption{Table of messages that a general \emph{Task} receives and
processes.}\tabularnewline
\toprule
\begin{minipage}[b]{0.29\columnwidth}\raggedright\strut
Message Subscription
\strut\end{minipage} &
\begin{minipage}[b]{0.65\columnwidth}\raggedright\strut
Description
\strut\end{minipage}\tabularnewline
\midrule
\endfirsthead
\toprule
\begin{minipage}[b]{0.29\columnwidth}\raggedright\strut
Message Subscription
\strut\end{minipage} &
\begin{minipage}[b]{0.65\columnwidth}\raggedright\strut
Description
\strut\end{minipage}\tabularnewline
\midrule
\endhead
\begin{minipage}[t]{0.29\columnwidth}\raggedright\strut
\emph{UniqueAutomationRequest}
\strut\end{minipage} &
\begin{minipage}[t]{0.65\columnwidth}\raggedright\strut
Indicates which \emph{Tasks} are to be considered as well as the set of
vehicles that can be used to fulfill those \emph{Tasks}. Upon reception
of this message, if a \emph{Task} ID is included, it will publish
\emph{TaskPlanOptions}.
\strut\end{minipage}\tabularnewline
\begin{minipage}[t]{0.29\columnwidth}\raggedright\strut
\begin{verbatim}
(2 ms work)
\end{verbatim}
\strut\end{minipage} &
\begin{minipage}[t]{0.65\columnwidth}\raggedright\strut
If included in the request, begin process of calculating task options
and costs by emitting \emph{SensorFootprintRequests} \textbf{Idle}
-\textgreater{} \textbf{SensorRequest} in normal operation
\textbf{OptionSelected} -\textgreater{} \textbf{SensorRequest}, when
interrupted
\strut\end{minipage}\tabularnewline
\begin{minipage}[t]{0.29\columnwidth}\raggedright\strut
\emph{TaskImplementationRequest}
\strut\end{minipage} &
\begin{minipage}[t]{0.65\columnwidth}\raggedright\strut
After an assignment has been made, each \emph{Task} involved is
requested to build the final set of waypoints that complete the
\emph{Task} and corresponding selected option. A \emph{Task} must build
the route \textbf{to} the \emph{Task} as well as waypoints that
implement the \emph{Task}. For each on-task waypoint, the
\emph{AssociatedTaskList} must include the \emph{Task} ID.
\strut\end{minipage}\tabularnewline
\begin{minipage}[t]{0.29\columnwidth}\raggedright\strut
\begin{verbatim}
(2 ms work)
\end{verbatim}
\strut\end{minipage} &
\begin{minipage}[t]{0.65\columnwidth}\raggedright\strut
\textbf{OptionsPublished} -\textgreater{} \textbf{FinalRoutes}, emit
\emph{RouteRequest} to determine final waypoint routes needed for
implementation
\strut\end{minipage}\tabularnewline
\begin{minipage}[t]{0.29\columnwidth}\raggedright\strut
\emph{EntityConfiguration}
\strut\end{minipage} &
\begin{minipage}[t]{0.65\columnwidth}\raggedright\strut
Vehicle capabilities (e.g.~allowable speeds) are described by entity
configuration messages. \emph{Tasks} can reason over sensor and vehicle
capabilites to present the proper options to other parts of the system.
If a vehicle does not have the capability to fulfill the \emph{Task}
(e.g.~does not have a proper sensor), then the \emph{Task} shall not
include that vehicle ID in the list of eligible entities reported as
part of an option.
\strut\end{minipage}\tabularnewline
\begin{minipage}[t]{0.29\columnwidth}\raggedright\strut
\begin{verbatim}
(0 ms work)
\end{verbatim}
\strut\end{minipage} &
\begin{minipage}[t]{0.65\columnwidth}\raggedright\strut
Add to internal storage for use in calculating options No state change
\strut\end{minipage}\tabularnewline
\begin{minipage}[t]{0.29\columnwidth}\raggedright\strut
\emph{EntityState}
\strut\end{minipage} &
\begin{minipage}[t]{0.65\columnwidth}\raggedright\strut
Describes the actual state of a vehicle in the system including
position, speed, and fuel status. This message is primary feedback
mechanism used for \emph{Tasks} to switch to an \textbf{Active} state.
When a \emph{Task} ID is listed in the \emph{AssociatedTaskList} of an
\emph{EntityState} message, the \emph{Task} is allowed to update
waypoints and sensor commands at will.
\strut\end{minipage}\tabularnewline
\begin{minipage}[t]{0.29\columnwidth}\raggedright\strut
\begin{verbatim}
(0 ms work)
\end{verbatim}
\strut\end{minipage} &
\begin{minipage}[t]{0.65\columnwidth}\raggedright\strut
\textbf{OptionSelected}: if \emph{Task} ID is listed in
\emph{AssociatedTaskList}, then -\textgreater{} \textbf{Active}
\textbf{Active}: if \emph{Task} ID is NOT listed in
\emph{AssociatedTaskList}, then -\textgreater{} \textbf{Completed}
\strut\end{minipage}\tabularnewline
\begin{minipage}[t]{0.29\columnwidth}\raggedright\strut
\emph{RouteResponse}
\strut\end{minipage} &
\begin{minipage}[t]{0.65\columnwidth}\raggedright\strut
Collection of route plans that fulfill a set of requests for navigation
through an \emph{OperatingRegion}. A \emph{Task} must request the
waypoints to route a vehicle from its last to the start of the
\emph{Task}. Additionally, this message can be used to obtain on-task
waypoints.
\strut\end{minipage}\tabularnewline
\begin{minipage}[t]{0.29\columnwidth}\raggedright\strut
\begin{verbatim}
(1 ms work)
\end{verbatim}
\strut\end{minipage} &
\begin{minipage}[t]{0.65\columnwidth}\raggedright\strut
\textbf{OptionRoutes} -\textgreater{} \textbf{OptionsPublished}, emit
\emph{TaskPlanOptions} \textbf{FinalRoutes} -\textgreater{}
\textbf{OptionSelected}, emit \emph{TaskImplementationResponse}
\strut\end{minipage}\tabularnewline
\begin{minipage}[t]{0.29\columnwidth}\raggedright\strut
\emph{SensorFootprintResponse}
\strut\end{minipage} &
\begin{minipage}[t]{0.65\columnwidth}\raggedright\strut
Collection of sensor information at different conditions corresponding
to a \emph{SensorFootprintRequests} message. Used to determine if a
particular entity with known sensor payloads can meet the sensor
resolution constraints required to fulfill this \emph{Task}.
\strut\end{minipage}\tabularnewline
\begin{minipage}[t]{0.29\columnwidth}\raggedright\strut
\begin{verbatim}
(1 ms work)
\end{verbatim}
\strut\end{minipage} &
\begin{minipage}[t]{0.65\columnwidth}\raggedright\strut
\textbf{SensorRequest} -\textgreater{} \textbf{OptionRoutes}, emit
\emph{RouteRequest} to determine on-\emph{Task} waypoints
\strut\end{minipage}\tabularnewline
\bottomrule
\end{longtable}

\begin{longtable}[c]{@{}ll@{}}
\caption{Table of messages that a general \emph{Task}
publishes.}\tabularnewline
\toprule
\begin{minipage}[b]{0.29\columnwidth}\raggedright\strut
Message Publication
\strut\end{minipage} &
\begin{minipage}[b]{0.65\columnwidth}\raggedright\strut
Description
\strut\end{minipage}\tabularnewline
\midrule
\endfirsthead
\toprule
\begin{minipage}[b]{0.29\columnwidth}\raggedright\strut
Message Publication
\strut\end{minipage} &
\begin{minipage}[b]{0.65\columnwidth}\raggedright\strut
Description
\strut\end{minipage}\tabularnewline
\midrule
\endhead
\begin{minipage}[t]{0.29\columnwidth}\raggedright\strut
\emph{TaskPlanOptions}
\strut\end{minipage} &
\begin{minipage}[t]{0.65\columnwidth}\raggedright\strut
Primary message published by a \emph{Task} to indicate the potential
different ways a \emph{Task} could be completed. Each possible way to
fulfill a \emph{Task} is listed as an \emph{option}. \emph{TaskOptions}
can also be related to each other via Process Algebra.
\strut\end{minipage}\tabularnewline
\begin{minipage}[t]{0.29\columnwidth}\raggedright\strut
\emph{TaskImplementationResponse}
\strut\end{minipage} &
\begin{minipage}[t]{0.65\columnwidth}\raggedright\strut
Primary message published by a \emph{Task} that reports the final set of
waypoints to both navigate the selected vehicle to the \emph{Task} as
well as the waypoints necessary to complete the \emph{Task} using the
selected option.
\strut\end{minipage}\tabularnewline
\begin{minipage}[t]{0.29\columnwidth}\raggedright\strut
\emph{RouteRequest}
\strut\end{minipage} &
\begin{minipage}[t]{0.65\columnwidth}\raggedright\strut
Collection of route plan requests to leverage the route planner
capability of constructing waypoints that adhere to the designated
\emph{OperatingRegion}. This request is made for waypoints en-route to
the \emph{Task} as well as on-task waypoints.
\strut\end{minipage}\tabularnewline
\begin{minipage}[t]{0.29\columnwidth}\raggedright\strut
\emph{SensorFootprintRequests}
\strut\end{minipage} &
\begin{minipage}[t]{0.65\columnwidth}\raggedright\strut
Collection of requests to the \emph{SensorManagerService} to determine
ground-sample distances possible for each potential entity. Uses camera
and gimbal information from the cached \emph{EntityConfiguration}
messages.
\strut\end{minipage}\tabularnewline
\begin{minipage}[t]{0.29\columnwidth}\raggedright\strut
\emph{VehicleActionCommand}
\strut\end{minipage} &
\begin{minipage}[t]{0.65\columnwidth}\raggedright\strut
When a \emph{Task} is \textbf{Active}, it is allowed to update sensor
navigation commands to on-task vehicles. This message is used to
directly command the vehicle to use the updated behaviors calculated by
the \emph{Task}.
\strut\end{minipage}\tabularnewline
\begin{minipage}[t]{0.29\columnwidth}\raggedright\strut
\emph{TaskComplete}
\strut\end{minipage} &
\begin{minipage}[t]{0.65\columnwidth}\raggedright\strut
Once a \emph{Task} has met its goal or if a vehicle reports that it is
no longer on-task, a previously \textbf{Active} \emph{Task} must send a
\emph{TaskComplete} message to inform the system of this change.
\strut\end{minipage}\tabularnewline
\bottomrule
\end{longtable}

\subsection{RoutePlannerVisibilityService}\label{routeplannervisibilityservice}

The \emph{RoutePlannerVisibilityService} is a service that provides
route planning using a visibility heuristic. One of the fundamental
architectural decisions in UxAS is separation of route planning from
task assignment. This service is an example of a route planning service
for aircraft. Ground vehicle route planning (based on Open Street Maps
data) can be found in the \emph{OsmPlannerService}.

The design of the \emph{RoutePlannerVisibilityService} message interface
is intended to be as simple as possible: a route planning service
considers routes only in fixed environments for known vehicles and
handles requests for single vehicles. The logic necessary to plan for
multiple (possibly heterogeneous) vehicles is handled in the
\emph{RouteAggregatorService}.

In two dimensional environments composed of polygons, the shortest
distance between points lies on the visibility graph. The
\emph{RoutePlannerVisibilityService} creates such a graph and, upon
request, adds desired start/end locations to quickly approximate a
distance-optimal route through the environment. With the straight-line
route created by the searching the visibility graph, a smoothing
operation is applied to ensure that minimum turn rate constraints of
vehicles are satisfied. Note, this smoothing operation can violate the
prescribed keep-out zones and is not guaranteed to smooth arbitrary
straight-line routes (in particular, path segments shorter than the
minimum turn radius can be problematic).

Due to the need to search over many possible orderings of \emph{Tasks}
during an assignment calculation, the route planner must very quickly
compute routes. Even for small problems, hundreds of routes must be
calculated before the assignment algorithm can start searching over the
possible ordering. For this reason it is imperative that the route
planner be responsive and efficient.

\begin{longtable}[c]{@{}ll@{}}
\caption{Table of messages that the \emph{RoutePlannerVisibilityService}
receives and processes.}\tabularnewline
\toprule
\begin{minipage}[b]{0.29\columnwidth}\raggedright\strut
Message Subscription
\strut\end{minipage} &
\begin{minipage}[b]{0.65\columnwidth}\raggedright\strut
Description
\strut\end{minipage}\tabularnewline
\midrule
\endfirsthead
\toprule
\begin{minipage}[b]{0.29\columnwidth}\raggedright\strut
Message Subscription
\strut\end{minipage} &
\begin{minipage}[b]{0.65\columnwidth}\raggedright\strut
Description
\strut\end{minipage}\tabularnewline
\midrule
\endhead
\begin{minipage}[t]{0.29\columnwidth}\raggedright\strut
\emph{RoutePlanRequest} ``AircraftPathPlanner''
\strut\end{minipage} &
\begin{minipage}[t]{0.65\columnwidth}\raggedright\strut
Primary message that describes a route plan request. A request considers
only a single vehicle in a single \emph{OperatingRegion} although it can
request multiple pairs of start and end locations with a single message.
In addition to subscribing to \emph{RoutePlanRequest}, this service also
subscribes to the group mailbox ``AircraftPathPlanner''. Upon reception
of a message on this channel, the service will process it as if it came
over the broadcast channel. The return message always uses
return-to-sender addressing.
\strut\end{minipage}\tabularnewline
\begin{minipage}[t]{0.29\columnwidth}\raggedright\strut
\begin{verbatim}
(10 ms work)
\end{verbatim}
\strut\end{minipage} &
\begin{minipage}[t]{0.65\columnwidth}\raggedright\strut
For each start/end pair, this service will compute a path that respects
the geometric constraints imposed by the corresponding
\emph{OperatingRegion}. For each start/end pair, the route planner could
reasonably work for 10ms to complete the calculation. Once all have been
calculated, emits \emph{RoutePlanResponse}.
\strut\end{minipage}\tabularnewline
\begin{minipage}[t]{0.29\columnwidth}\raggedright\strut
\emph{KeepOutZone}
\strut\end{minipage} &
\begin{minipage}[t]{0.65\columnwidth}\raggedright\strut
Polygon description of a region in which vehicles must not travel. This
service will track all \emph{KeepOutZones} to compose them upon
reception of an \emph{OperatingRegion}.
\strut\end{minipage}\tabularnewline
\begin{minipage}[t]{0.29\columnwidth}\raggedright\strut
\begin{verbatim}
(0 ms work)
\end{verbatim}
\strut\end{minipage} &
\begin{minipage}[t]{0.65\columnwidth}\raggedright\strut
Store for use during calculation of operating region map
\strut\end{minipage}\tabularnewline
\begin{minipage}[t]{0.29\columnwidth}\raggedright\strut
\emph{KeepInZone}
\strut\end{minipage} &
\begin{minipage}[t]{0.65\columnwidth}\raggedright\strut
Polygon description of a region in which vehicles must remain during
travel. This service will track all \emph{KeepInZones} to compose them
upon reception of an \emph{OperatingRegion}.
\strut\end{minipage}\tabularnewline
\begin{minipage}[t]{0.29\columnwidth}\raggedright\strut
\begin{verbatim}
(0 ms work)
\end{verbatim}
\strut\end{minipage} &
\begin{minipage}[t]{0.65\columnwidth}\raggedright\strut
Store for use during calculation of operating region map
\strut\end{minipage}\tabularnewline
\begin{minipage}[t]{0.29\columnwidth}\raggedright\strut
\emph{OperatingRegion}
\strut\end{minipage} &
\begin{minipage}[t]{0.65\columnwidth}\raggedright\strut
Collection of \emph{KeepIn} and \emph{KeepOut} zones that describe the
allowable space for vehicular travel. When received, this service
creates a visibility graph considering the zones referenced by this
\emph{OperatingRegion}. Upon \emph{RoutePlanRequest} the visibility
graph corresponding to the \emph{OperatingRegion} ID is retreived and
manipulated to add start/end locations and perform the shortest path
search.
\strut\end{minipage}\tabularnewline
\begin{minipage}[t]{0.29\columnwidth}\raggedright\strut
\begin{verbatim}
(20 ms work)
\end{verbatim}
\strut\end{minipage} &
\begin{minipage}[t]{0.65\columnwidth}\raggedright\strut
To respond quickly to \emph{RoutePlanRequest} messages, the
\emph{RoutePlannerVisibilityService} will create a pre-processed map
using the geometric constraints from the \emph{OperatingRegion}. The
result is stored for later requests.
\strut\end{minipage}\tabularnewline
\begin{minipage}[t]{0.29\columnwidth}\raggedright\strut
\emph{EntityConfiguration}
\strut\end{minipage} &
\begin{minipage}[t]{0.65\columnwidth}\raggedright\strut
Vehicle capabilities (e.g.~allowable speeds) are described by entity
configuration messages. This service calculates the minimum turn radius
of the entity by using the max bank angle and nominal speed. Requested
routes are then returned at the nominal speed and with turns
approximating the minimum turn radius.
\strut\end{minipage}\tabularnewline
\begin{minipage}[t]{0.29\columnwidth}\raggedright\strut
\begin{verbatim}
(20 ms work)
\end{verbatim}
\strut\end{minipage} &
\begin{minipage}[t]{0.65\columnwidth}\raggedright\strut
Stored for use during \emph{RoutePlanRequest}s. Also used to update
operating region maps based on the capability of the entity.
\strut\end{minipage}\tabularnewline
\bottomrule
\end{longtable}

\begin{longtable}[c]{@{}ll@{}}
\caption{Table of messages that the \emph{RoutePlannerVisibilityService}
publishes.}\tabularnewline
\toprule
\begin{minipage}[b]{0.29\columnwidth}\raggedright\strut
Message Publication
\strut\end{minipage} &
\begin{minipage}[b]{0.65\columnwidth}\raggedright\strut
Description
\strut\end{minipage}\tabularnewline
\midrule
\endfirsthead
\toprule
\begin{minipage}[b]{0.29\columnwidth}\raggedright\strut
Message Publication
\strut\end{minipage} &
\begin{minipage}[b]{0.65\columnwidth}\raggedright\strut
Description
\strut\end{minipage}\tabularnewline
\midrule
\endhead
\begin{minipage}[t]{0.29\columnwidth}\raggedright\strut
\emph{RoutePlanResponse}
\strut\end{minipage} &
\begin{minipage}[t]{0.65\columnwidth}\raggedright\strut
This message contains the waypoints and time cost that fulfills the
route request. This message is the only one published by the
\emph{RoutePlannerVisibilityService} and is always sent using the
return-to-sender addressing which ensures that only the original
requester receives the response.
\strut\end{minipage}\tabularnewline
\bottomrule
\end{longtable}

\subsection{RouteAggregatorService}\label{routeaggregatorservice}

The \emph{RouteAggregatorService} fills two primary roles: 1) it acts as
a helper service to make route requests for large numbers of
heterogenous vehicles; and 2) it constructs the task-to-task route-cost
table that is used by the assignment service to order the tasks as
efficiently as possible. Each functional role acts independently and can
be modeled as two different state machines.

The \emph{Aggregator} role orchestrates large numbers of route requests
(possibly to multiple route planners). This allows other services in the
system (such as \emph{Tasks}) to make a single request for routes and
receive a single reply with the complete set of routes for numerous
vehicles.

For every aggregate route request (specified by a \emph{RouteRequest}
message), the \emph{Aggregator} makes a series of
\emph{RoutePlanRequests} to the appropriate route planners (i.e.~sending
route plan requests for ground vehicles to the ground vehicle planner
and route plan requests for aircraft to the aircraft planner). Each
request is marked with a request ID and a list of all request IDs that
must have matching replies is created. The \emph{Aggregator} then enters
a \textbf{pending} state in which all received plan replies are stored
and then checked off the list of expected replies. When all of the
expected replies have been received, the \emph{Aggregator} publishes the
completed \emph{RouteResponse} and returns to the \textbf{idle} state.

Note that every aggregate route request corresponds to a separate
internal checklist of expected responses that will fulfill the original
aggregate request. The \emph{Aggregator} is designed to service each
aggregate route request even if a previous one is in the process of
being fulfilled. When the \emph{Aggregator} receives any response from a
route planner, it checks each of the many checklists to determine if all
expected responses for a particuarl list have been met. In this way, the
\emph{Aggregator} is in a different \textbf{pending} state for each
aggregate request made to it.

\begin{longtable}[c]{@{}ll@{}}
\caption{Table of messages that the \emph{RouteAggregatorService}
receives and processes in its \emph{Aggregator} role.}\tabularnewline
\toprule
\begin{minipage}[b]{0.29\columnwidth}\raggedright\strut
Message Subscription
\strut\end{minipage} &
\begin{minipage}[b]{0.65\columnwidth}\raggedright\strut
Description
\strut\end{minipage}\tabularnewline
\midrule
\endfirsthead
\toprule
\begin{minipage}[b]{0.29\columnwidth}\raggedright\strut
Message Subscription
\strut\end{minipage} &
\begin{minipage}[b]{0.65\columnwidth}\raggedright\strut
Description
\strut\end{minipage}\tabularnewline
\midrule
\endhead
\begin{minipage}[t]{0.29\columnwidth}\raggedright\strut
\emph{RouteRequest}
\strut\end{minipage} &
\begin{minipage}[t]{0.65\columnwidth}\raggedright\strut
Primary message that requests a large number of routes for potentially
heterogeneous vehicles. The \emph{Aggregator} will make a series of
\emph{RoutePlanRequests} to the appropriate planners to fulfill this
request.
\strut\end{minipage}\tabularnewline
\begin{minipage}[t]{0.29\columnwidth}\raggedright\strut
\begin{verbatim}
(1 ms work)
\end{verbatim}
\strut\end{minipage} &
\begin{minipage}[t]{0.65\columnwidth}\raggedright\strut
\textbf{idle} -\textgreater{} \textbf{pending}, indexed by
\emph{RouteRequest} request ID, create a checklist of expected
responses. Emit a number of \emph{RoutePlanRequest} messages equal to
the number of vehicles in the \texttt{VehicleID} field of the original
\emph{RouteRequest}
\strut\end{minipage}\tabularnewline
\begin{minipage}[t]{0.29\columnwidth}\raggedright\strut
\emph{EntityConfiguration}
\strut\end{minipage} &
\begin{minipage}[t]{0.65\columnwidth}\raggedright\strut
Vehicle capabilities (e.g.~allowable speeds) are described by entity
configuration messages. This service uses the \emph{EntityConfiguration}
to determine which type of vehicle corresponds to a specific ID so that
ground planners are used for ground vehicles and air planners are used
for aircraft.
\strut\end{minipage}\tabularnewline
\begin{minipage}[t]{0.29\columnwidth}\raggedright\strut
\begin{verbatim}
(0 ms work)
\end{verbatim}
\strut\end{minipage} &
\begin{minipage}[t]{0.65\columnwidth}\raggedright\strut
No state change. Store to identify appropriate route planner by vehicle
ID.
\strut\end{minipage}\tabularnewline
\begin{minipage}[t]{0.29\columnwidth}\raggedright\strut
\emph{RoutePlanResponse}
\strut\end{minipage} &
\begin{minipage}[t]{0.65\columnwidth}\raggedright\strut
This message is the fulfillment of a single vehicle route plan request
which the \emph{Aggregator} catalogues until the complete set of
expected responses is received.
\strut\end{minipage}\tabularnewline
\begin{minipage}[t]{0.29\columnwidth}\raggedright\strut
\begin{verbatim}
(1 ms work)
\end{verbatim}
\strut\end{minipage} &
\begin{minipage}[t]{0.65\columnwidth}\raggedright\strut
Store response and check to see if this message completes any checklist.
If a checklist is complete, use the corresponding request ID to create a
complete \emph{RouteResponse} message. Emit \emph{RouteResponse} and
\textbf{pending} -\textgreater{} \textbf{idle}.
\strut\end{minipage}\tabularnewline
\bottomrule
\end{longtable}

\begin{longtable}[c]{@{}ll@{}}
\caption{Table of messages that the \emph{RouteAggregatorService}
publishes in its \emph{Aggregator} role.}\tabularnewline
\toprule
\begin{minipage}[b]{0.29\columnwidth}\raggedright\strut
Message Publication
\strut\end{minipage} &
\begin{minipage}[b]{0.65\columnwidth}\raggedright\strut
Description
\strut\end{minipage}\tabularnewline
\midrule
\endfirsthead
\toprule
\begin{minipage}[b]{0.29\columnwidth}\raggedright\strut
Message Publication
\strut\end{minipage} &
\begin{minipage}[b]{0.65\columnwidth}\raggedright\strut
Description
\strut\end{minipage}\tabularnewline
\midrule
\endhead
\begin{minipage}[t]{0.29\columnwidth}\raggedright\strut
\emph{RouteResponse}
\strut\end{minipage} &
\begin{minipage}[t]{0.65\columnwidth}\raggedright\strut
Once the \emph{Aggregator} has a complete set of responses collected
from the route planners, the message is built as a reply to the original
\emph{RouteRequest}.
\strut\end{minipage}\tabularnewline
\begin{minipage}[t]{0.29\columnwidth}\raggedright\strut
\emph{RoutePlanRequest}
\strut\end{minipage} &
\begin{minipage}[t]{0.65\columnwidth}\raggedright\strut
The \emph{Aggregator} publishes a series of these requests in order to
fulfill an aggregate route request. These messages are published in
batch, without waiting for a reply. It is expected that eventually all
requests made will be fulfilled.
\strut\end{minipage}\tabularnewline
\bottomrule
\end{longtable}

The \emph{RouteAggregatorService} also acts in the role of creating the
\emph{AssignmentCostMatrix} which is a key input to the assignment
service. For simplicity, this role will be labeled as the
\emph{Collector} role. This role is triggered by the
\emph{UniqueAutomationRequest} message and begins the process of
collecting a complete set of on-task and between-task costs.

The \emph{Collector} starts in the \textbf{Idle} state and upon
reception of a \emph{UniqueAutomationRequest} message, it creates a list
of \emph{Task} IDs that are involved in the request and then moves to
the \textbf{OptionsWait} state. In this state, the \emph{Collector}
stores all \emph{TaskPlanOptions} and matches them to the IDs of the
\emph{Task} IDs that were requested in the
\emph{UniqueAutomationRequest}. When the expected list of \emph{Tasks}
is associated with a corresponding \emph{TaskPlanOptions}, the
\emph{Collector} moves to the \textbf{RoutePending} state. In this
state, the \emph{Collector} makes a series of route plan requests from
1) initial conditions of all vehicles to all tasks and 2) route plans
between the end of each \emph{Task} and start of all other \emph{Tasks}.
Similar to the \emph{Aggregator}, the \emph{Collector} creates a
checklist of expected route plan responses and uses that checklist to
determine when the complete set of routes has been returned from the
route planners. The \emph{Collector} remains in the
\textbf{RoutePending} state until all route requests have been
fulfilled, at which point it collates the responses into a complete
\emph{AssignmentCostMatrix}. The \emph{AssignmentCostMatrix} message is
published and the \emph{Collector} returns to the \textbf{Idle} state.

Note that the \emph{AutomationValidatorService} ensures that only a
single \emph{UniqueAutomationRequest} is handled by the system at a
time. However, the design of the \emph{Collector} does allow for
multiple simultaneous requests as all checklists (for pending route and
task option messages) are associated with the unique ID from each
\emph{UniqueAutomationRequest}.

\begin{longtable}[c]{@{}ll@{}}
\caption{Table of messages that the \emph{RouteAggregatorService}
receives and processes in its \emph{Collector} role.}\tabularnewline
\toprule
\begin{minipage}[b]{0.29\columnwidth}\raggedright\strut
Message Subscription
\strut\end{minipage} &
\begin{minipage}[b]{0.65\columnwidth}\raggedright\strut
Description
\strut\end{minipage}\tabularnewline
\midrule
\endfirsthead
\toprule
\begin{minipage}[b]{0.29\columnwidth}\raggedright\strut
Message Subscription
\strut\end{minipage} &
\begin{minipage}[b]{0.65\columnwidth}\raggedright\strut
Description
\strut\end{minipage}\tabularnewline
\midrule
\endhead
\begin{minipage}[t]{0.29\columnwidth}\raggedright\strut
\emph{UniqueAutomationRequest}
\strut\end{minipage} &
\begin{minipage}[t]{0.65\columnwidth}\raggedright\strut
Primary message that initiates the collection of options sent from each
\emph{Task} via the \emph{TaskPlanOptions} message. A list of all
\emph{Tasks} included in the \emph{UniqueAutomationRequest} is made upon
reception of this message and later used to ensure that all included
\emph{Tasks} have responded.
\strut\end{minipage}\tabularnewline
\begin{minipage}[t]{0.29\columnwidth}\raggedright\strut
\begin{verbatim}
(1 ms work)
\end{verbatim}
\strut\end{minipage} &
\begin{minipage}[t]{0.65\columnwidth}\raggedright\strut
\textbf{Idle} -\textgreater{} \textbf{OptionsWait}, create checklist of
expected task options.
\strut\end{minipage}\tabularnewline
\begin{minipage}[t]{0.29\columnwidth}\raggedright\strut
\emph{TaskPlanOptions}
\strut\end{minipage} &
\begin{minipage}[t]{0.65\columnwidth}\raggedright\strut
Primary message from \emph{Tasks} that prescribe available start and end
locations for each option as well as cost to complete the option. Once
all expected \emph{TaskPlanOptions} have been received, the
\emph{Collector} will use the current locations of the vehicles to
request paths from each vehicle to each task option and from each task
option to every other task option.
\strut\end{minipage}\tabularnewline
\begin{minipage}[t]{0.29\columnwidth}\raggedright\strut
\begin{verbatim}
(1 ms work)
\end{verbatim}
\strut\end{minipage} &
\begin{minipage}[t]{0.65\columnwidth}\raggedright\strut
Store task options and check to see if this message completes the
checklist. If the checklist is complete, create a series of
\emph{RoutePlanRequest} messages to find routes from the current
locations of vehicles to each task and from each task to every other
task. Emit this series of \emph{RoutePlanRequest} messages,
\textbf{OptionsWait} -\textgreater{} \textbf{RoutePending}.
\strut\end{minipage}\tabularnewline
\begin{minipage}[t]{0.29\columnwidth}\raggedright\strut
\emph{EntityState}
\strut\end{minipage} &
\begin{minipage}[t]{0.65\columnwidth}\raggedright\strut
Describes the actual state of a vehicle in the system including
position, speed, and fuel status. This message is used to create routes
and cost estimates from the associated vehicle position and heading to
the task option start locations.
\strut\end{minipage}\tabularnewline
\begin{minipage}[t]{0.29\columnwidth}\raggedright\strut
\begin{verbatim}
(0 ms work)
\end{verbatim}
\strut\end{minipage} &
\begin{minipage}[t]{0.65\columnwidth}\raggedright\strut
No state change. Store for use in requesting routes from vehicle
positions to task start locations.
\strut\end{minipage}\tabularnewline
\begin{minipage}[t]{0.29\columnwidth}\raggedright\strut
\emph{RoutePlanResponse}
\strut\end{minipage} &
\begin{minipage}[t]{0.65\columnwidth}\raggedright\strut
This message is the fulfillment of a single vehicle route plan request
which the \emph{Collector} catalogues until the complete set of expected
responses is received.
\strut\end{minipage}\tabularnewline
\begin{minipage}[t]{0.29\columnwidth}\raggedright\strut
\begin{verbatim}
(1 ms work)
\end{verbatim}
\strut\end{minipage} &
\begin{minipage}[t]{0.65\columnwidth}\raggedright\strut
Store response and check to see if this message completes the cost
matrix. If so, emit \emph{AssignmentCostMatrix} and
\textbf{RoutePending} -\textgreater{} \textbf{Idle}.
\strut\end{minipage}\tabularnewline
\bottomrule
\end{longtable}

\begin{longtable}[c]{@{}ll@{}}
\caption{Table of messages that the \emph{RouteAggregatorService}
publishes in its \emph{Collector} role.}\tabularnewline
\toprule
\begin{minipage}[b]{0.29\columnwidth}\raggedright\strut
Message Publication
\strut\end{minipage} &
\begin{minipage}[b]{0.65\columnwidth}\raggedright\strut
Description
\strut\end{minipage}\tabularnewline
\midrule
\endfirsthead
\toprule
\begin{minipage}[b]{0.29\columnwidth}\raggedright\strut
Message Publication
\strut\end{minipage} &
\begin{minipage}[b]{0.65\columnwidth}\raggedright\strut
Description
\strut\end{minipage}\tabularnewline
\midrule
\endhead
\begin{minipage}[t]{0.29\columnwidth}\raggedright\strut
\emph{AssignmentCostMatrix}
\strut\end{minipage} &
\begin{minipage}[t]{0.65\columnwidth}\raggedright\strut
Once the \emph{Collector} has a complete set of \emph{TaskPlanOptions}
as well as routes between tasks and vehicles, this message is built to
inform the next step in the task assignment pipeline: the
\emph{AssignmentTreeBranchBoundService}.
\strut\end{minipage}\tabularnewline
\begin{minipage}[t]{0.29\columnwidth}\raggedright\strut
\emph{RoutePlanRequest}
\strut\end{minipage} &
\begin{minipage}[t]{0.65\columnwidth}\raggedright\strut
The \emph{Collector} publishes a series of these requests in order to
compute the vehicle-to-task and task-to-task route costs. These messages
are published in batch, without waiting for a reply. It is expected that
eventually all requests made will be fulfilled.
\strut\end{minipage}\tabularnewline
\bottomrule
\end{longtable}

\subsection{AssignmentTreeBranchBoundService}\label{assignmenttreebranchboundservice}

The \emph{AssignmentTreeBranchBoundService} is a service that does the
primary computation to determine an efficient ordering and assignment of
all \emph{Tasks} to the available vehicles. The assignment algorithm
reasons only at the cost level; in other words, the assignment itself
does not directly consider vehicle motion but rather it uses estimates
of that motion cost. The cost estimates are provided by the \emph{Tasks}
(for on-task costs) and by the \emph{RouteAggregatorService} for
task-to-task travel costs.

The \emph{AssignmentTreeBranchBoundService} can be configured to
optimize based on cumulative team cost (i.e.~sum total of time required
from each vehicle) or the maximium time of final task completion
(i.e.~only the final time of total mission completion is minimized). For
either optimization type, this service will first find a feasible
solution by executing a depth-first, greedy search. Although it is
possible to request a mission for which \textbf{no} feasible solution
exists, the vast majority of missions are underconstrained and have an
exponential (relative to numbers of vehicles and tasks) number of
solutions from which an efficient one must be discovered.

After the \emph{AssignmentTreeBranchBoundService} obtains a greedy
solution to the assignment problem, it will continue to search the space
of possibilities via backtracking up the tree of possibilities and
\emph{branching} at descision points. The cost of the greedy solution
acts as a \emph{bound} beyond which no solution is be considered. In
other words, as more efficient solutions are discovered, any partial
solution that exeeds the cost of the current best solution will
immediately be abandoned (cut) to focus search effort in the part of the
space that could possibly lead to better solutions. In this way,
solution search progresses until all possibilities have been exhausted
or a pre-determined tree size has been searched. By placing an upper
limit on the size of the tree to search, worst-case bounds on
computation time can be made to ensure desired responsiveness from the
\emph{AssignmentTreeBranchBoundService}.

General assignment problems do not normally allow for specification of
\emph{Task} relationships. However, the
\emph{AssignmentTreeBranchBoundService} relies on the ability to specify
\emph{Task} relationships via Process Algebra constraints. This enables
creation of moderately complex missions from simple atomic \emph{Tasks}.
Adherence to Process Algebra constraints also allows \emph{Tasks} to
describe their \emph{option} relationships. The Process Algebra
relationships of a particular \emph{Task} option are directly
substituted into and replace the original \emph{Task} in the
mission-level Process Algebra specification. Due to the heavy reliance
on Process Algebra specifications, any assignment service that replaces
\emph{AssignmentTreeBranchBoundService} must also guarantee satisfaction
of such specifications.

The behavior of the \emph{AssignmentTreeBranchBoundService} is
straight-foward. Upon reception of a \emph{UniqueAutomationRequest},
this service enters the \textbf{wait} state and remains in this state
until a complete set of \emph{TaskPlanOptions} and an
\emph{AssignmentCostMatrix} message have been received. In the
\textbf{wait} state, a running list of the expected
\emph{TaskPlanOptions} is maintained and checked off when received. Upon
receiving the \emph{AssignmentCostMatrix} (which should be received
strictly after the \emph{TaskPlanOptions} due to the behavior of the
\emph{RouteAggregatorService}), this service conducts the
branch-and-bound search to determine the proper ordering and assignment
of \emph{Tasks} to vehicles. The results of the optimization are
packaged into the \emph{TaskAssignmentSummary} and published, at which
point this service returns to the \textbf{idle} state.

\begin{longtable}[c]{@{}ll@{}}
\caption{Table of messages that the
\emph{AssignmentTreeBranchBoundService} receives and
processes.}\tabularnewline
\toprule
\begin{minipage}[b]{0.29\columnwidth}\raggedright\strut
Message Subscription
\strut\end{minipage} &
\begin{minipage}[b]{0.65\columnwidth}\raggedright\strut
Description
\strut\end{minipage}\tabularnewline
\midrule
\endfirsthead
\toprule
\begin{minipage}[b]{0.29\columnwidth}\raggedright\strut
Message Subscription
\strut\end{minipage} &
\begin{minipage}[b]{0.65\columnwidth}\raggedright\strut
Description
\strut\end{minipage}\tabularnewline
\midrule
\endhead
\begin{minipage}[t]{0.29\columnwidth}\raggedright\strut
\emph{UniqueAutomationRequest}
\strut\end{minipage} &
\begin{minipage}[t]{0.65\columnwidth}\raggedright\strut
Sentinel message that initiates the collection of options sent from each
\emph{Task} via the \emph{TaskPlanOptions} message. A list of all
\emph{Tasks} included in the \emph{UniqueAutomationRequest} is made upon
reception of this message and later used to ensure that all included
\emph{Tasks} have responded.
\strut\end{minipage}\tabularnewline
\begin{minipage}[t]{0.29\columnwidth}\raggedright\strut
\begin{verbatim}
(0 ms work)
\end{verbatim}
\strut\end{minipage} &
\begin{minipage}[t]{0.65\columnwidth}\raggedright\strut
\textbf{idle} -\textgreater{} \textbf{wait}, store request ID for
identification of corresponding \emph{TaskPlanOptions} and
\emph{AssignmentCostMatrix}.
\strut\end{minipage}\tabularnewline
\begin{minipage}[t]{0.29\columnwidth}\raggedright\strut
\emph{TaskPlanOptions}
\strut\end{minipage} &
\begin{minipage}[t]{0.65\columnwidth}\raggedright\strut
Primary message from \emph{Tasks} that prescribe available start and end
locations for each option as well as cost to complete the option. In the
\textbf{wait} state, this service will store all reported options for
use in calculating mission cost for vehicles when considering possible
assignments.
\strut\end{minipage}\tabularnewline
\begin{minipage}[t]{0.29\columnwidth}\raggedright\strut
\begin{verbatim}
(0 ms work)
\end{verbatim}
\strut\end{minipage} &
\begin{minipage}[t]{0.65\columnwidth}\raggedright\strut
No state change. Store cost of each task option for look-up during
optimization.
\strut\end{minipage}\tabularnewline
\begin{minipage}[t]{0.29\columnwidth}\raggedright\strut
\emph{AssignmentCostMatrix}
\strut\end{minipage} &
\begin{minipage}[t]{0.65\columnwidth}\raggedright\strut
Primary message that initiates the task assignment optimization. This
message contains the task-to-task routing cost estimates and is a key
factor in determining which vehicle could most efficiently reach a
\emph{Task}. Coupled with the on-task costs captured in the
\emph{TaskPlanOptions}, a complete reasoning over both traveling to and
completing a \emph{Task} can be looked up during the search over
possible \emph{Task} orderings.
\strut\end{minipage}\tabularnewline
\begin{minipage}[t]{0.29\columnwidth}\raggedright\strut
\begin{verbatim}
(1500 ms work)
\end{verbatim}
\strut\end{minipage} &
\begin{minipage}[t]{0.65\columnwidth}\raggedright\strut
Using the cost of each task option (from the stored
\emph{TaskPlanOptions} messages) and the cost for each vehicle to reach
each option (from \emph{AssignmentCostMatrix}), perform an optimization
attempting to find the minimal cost mission that adheres to the Process
Algebra contraints. Upon completion, emit \emph{TaskAssignmentSummary}
and \textbf{wait} -\textgreater{} \textbf{idle}.
\strut\end{minipage}\tabularnewline
\bottomrule
\end{longtable}

\begin{longtable}[c]{@{}ll@{}}
\caption{Table of messages that the
\emph{AssignmentTreeBranchBoundService} publishes.}\tabularnewline
\toprule
\begin{minipage}[b]{0.29\columnwidth}\raggedright\strut
Message Publication
\strut\end{minipage} &
\begin{minipage}[b]{0.65\columnwidth}\raggedright\strut
Description
\strut\end{minipage}\tabularnewline
\midrule
\endfirsthead
\toprule
\begin{minipage}[b]{0.29\columnwidth}\raggedright\strut
Message Publication
\strut\end{minipage} &
\begin{minipage}[b]{0.65\columnwidth}\raggedright\strut
Description
\strut\end{minipage}\tabularnewline
\midrule
\endhead
\begin{minipage}[t]{0.29\columnwidth}\raggedright\strut
\emph{TaskAssignmentSummary}
\strut\end{minipage} &
\begin{minipage}[t]{0.65\columnwidth}\raggedright\strut
The singular message published by this service which precisely describes
the proper ordering of \emph{Tasks} and the vehicles that are assigned
to complete each \emph{Task}.
\strut\end{minipage}\tabularnewline
\bottomrule
\end{longtable}

\subsection{PlanBuilderService}\label{planbuilderservice}

The final step in the task assignment pipeline is converting the
decisions made by the \emph{AssignmentTreeBranchBoundService} into
waypoint paths that can be sent to each of the vehicles. Using the
ordering of \emph{Tasks} and the assigned vehicle(s) for each
\emph{Task}, the \emph{PlanBuilderService} will query each \emph{Task}
in turn to construct enroute and on-task waypoints to complete the
mission.

Similar to both the \emph{RouteAggregator} and the
\emph{AssignmentTreeBranchBoundService}, the \emph{PlanBuilderService}
utilizes a received \emph{UniqueAutomationRequest} to detect that a new
mission request has been made to the system. The
\emph{UniqueAutomationRequest} is stored until a
\emph{TaskAssignmentSummary} that corresponds to the unique ID is
received. At this point, the \emph{PlanBuilderService} transitions from
the \textbf{idle} state to the \textbf{busy} state.

Using the list of ordered \emph{Tasks} dictated by the
\emph{TaskAssignmentSummary}, the \emph{PlanBuilderService} sends a
\emph{TaskImplementationRequest} to each \emph{Task} in order and waits
for a \emph{TaskImplementationResponse} from each \emph{Task} before
moving to the next. This is necessary as the ending location of a
previous \emph{Task} becomes the starting location for a subsequent
\emph{Task}. Since each \emph{Task} is allowed to refine its final
waypoint plan at this stage, the exact ending location may be different
that was was originally indicated during the \emph{TaskPlanOptions}
phase. By working through the \emph{Task} list in assignment order, all
uncertainty about timing and location is elminated and each \emph{Task}
is allowed to make a final determination on the waypoints to be used.

Once all \emph{Tasks} have reponded with a
\emph{TaskImplementationResponse}, the \emph{PlanBuilderService} links
all waypoints for each vehicle into a complete \emph{MissionCommand}.
The total set of \emph{MissionCommands} are collected into the
\emph{UniqueAutomationResponse} which is broadcast to the system and
represents a complete solution to the original \emph{AutomationRequest}.
At this point, the \emph{PlanBuilderService} returns to the
\textbf{idle} state.

\begin{longtable}[c]{@{}ll@{}}
\caption{Table of messages that the \emph{PlanBuilderService} receives
and processes.}\tabularnewline
\toprule
\begin{minipage}[b]{0.29\columnwidth}\raggedright\strut
Message Subscription
\strut\end{minipage} &
\begin{minipage}[b]{0.65\columnwidth}\raggedright\strut
Description
\strut\end{minipage}\tabularnewline
\midrule
\endfirsthead
\toprule
\begin{minipage}[b]{0.29\columnwidth}\raggedright\strut
Message Subscription
\strut\end{minipage} &
\begin{minipage}[b]{0.65\columnwidth}\raggedright\strut
Description
\strut\end{minipage}\tabularnewline
\midrule
\endhead
\begin{minipage}[t]{0.29\columnwidth}\raggedright\strut
\emph{TaskAssignmentSummary}
\strut\end{minipage} &
\begin{minipage}[t]{0.65\columnwidth}\raggedright\strut
Primary message that dictates the proper order and vehicle assignment to
efficiently carry out the requested mission. Upon reception of this
messsage, the \emph{PlanBuilderService} queries each \emph{Task} in
order for the final waypoint paths.
\strut\end{minipage}\tabularnewline
\begin{minipage}[t]{0.29\columnwidth}\raggedright\strut
\begin{verbatim}
(2 ms work)
\end{verbatim}
\strut\end{minipage} &
\begin{minipage}[t]{0.65\columnwidth}\raggedright\strut
\textbf{idle} -\textgreater{} \textbf{busy}, create a queue of ordered
\emph{TaskImplementationRequest} messages in the order prescribed by the
\emph{TaskAssignmentSummary}. Emit request at top of queue.
\strut\end{minipage}\tabularnewline
\begin{minipage}[t]{0.29\columnwidth}\raggedright\strut
\emph{EntityState}
\strut\end{minipage} &
\begin{minipage}[t]{0.65\columnwidth}\raggedright\strut
Describes the actual state of a vehicle in the system including
position, speed, and fuel status. This message is used to inform the
first \emph{Task} of the location of the vehicles. Subsequent
\emph{Tasks} use the predicted positions and headings of vehicles after
previous \emph{Tasks} have reported waypoints earlier in the mission.
\strut\end{minipage}\tabularnewline
\begin{minipage}[t]{0.29\columnwidth}\raggedright\strut
\begin{verbatim}
(0 ms work)
\end{verbatim}
\strut\end{minipage} &
\begin{minipage}[t]{0.65\columnwidth}\raggedright\strut
No state change. Store for use in creating
\emph{TaskImplementationRequest} messages.
\strut\end{minipage}\tabularnewline
\begin{minipage}[t]{0.29\columnwidth}\raggedright\strut
\emph{TaskImplementationResponse}
\strut\end{minipage} &
\begin{minipage}[t]{0.65\columnwidth}\raggedright\strut
Primary message that each \emph{Task} reports to inform this service of
the precise waypoints that need to be followed to reach the \emph{Task}
and carry it out correctly. The ordered collection of these messages are
used to build the final \emph{UniqueAutomationResponse}.
\strut\end{minipage}\tabularnewline
\begin{minipage}[t]{0.29\columnwidth}\raggedright\strut
\begin{verbatim}
(2 ms work)
\end{verbatim}
\strut\end{minipage} &
\begin{minipage}[t]{0.65\columnwidth}\raggedright\strut
Remove top of task request queue and update predicted locations of
vehicles. If task request queue is not empty, configure request at top
of queue with predicted vehicle positions and emit the corresponding
\emph{TaskImplementationRequest} message. If queue is empty,
\textbf{busy} -\textgreater{} \textbf{idle}.
\strut\end{minipage}\tabularnewline
\begin{minipage}[t]{0.29\columnwidth}\raggedright\strut
\emph{UniqueAutomationRequest}
\strut\end{minipage} &
\begin{minipage}[t]{0.65\columnwidth}\raggedright\strut
Informs this service of a new mission request in the system. Contains
the desired starting locations and headings of the vehicles that are to
be considered as part of the solution.
\strut\end{minipage}\tabularnewline
\begin{minipage}[t]{0.29\columnwidth}\raggedright\strut
\begin{verbatim}
(0 ms work)
\end{verbatim}
\strut\end{minipage} &
\begin{minipage}[t]{0.65\columnwidth}\raggedright\strut
No state change. Store for use in creating
\emph{TaskImplementationRequest} messages. Note, positions in
\emph{UniqueAutomationRequest} override reported state positions stored
when \emph{EntityState} messages are received.
\strut\end{minipage}\tabularnewline
\bottomrule
\end{longtable}

\begin{longtable}[c]{@{}ll@{}}
\caption{Table of messages that the \emph{PlanBuilderService}
publishes.}\tabularnewline
\toprule
\begin{minipage}[b]{0.29\columnwidth}\raggedright\strut
Message Publication
\strut\end{minipage} &
\begin{minipage}[b]{0.65\columnwidth}\raggedright\strut
Description
\strut\end{minipage}\tabularnewline
\midrule
\endfirsthead
\toprule
\begin{minipage}[b]{0.29\columnwidth}\raggedright\strut
Message Publication
\strut\end{minipage} &
\begin{minipage}[b]{0.65\columnwidth}\raggedright\strut
Description
\strut\end{minipage}\tabularnewline
\midrule
\endhead
\begin{minipage}[t]{0.29\columnwidth}\raggedright\strut
\emph{TaskImplementationRequest}
\strut\end{minipage} &
\begin{minipage}[t]{0.65\columnwidth}\raggedright\strut
The primary message used to query each \emph{Task} for the proper
waypoints that both reach and carry out the \emph{Task}. Once the
\emph{PlanBuilderService} receives a corresponding response from each
\emph{Task}, it can construct a final set of waypoints for each vehicle.
\strut\end{minipage}\tabularnewline
\begin{minipage}[t]{0.29\columnwidth}\raggedright\strut
\emph{UniqueAutomationResponse}
\strut\end{minipage} &
\begin{minipage}[t]{0.65\columnwidth}\raggedright\strut
This message contains a list of waypoints for each vehicle that was
considered during the automation request. This collection of complete
waypoints for the team fulfills the original request.
\strut\end{minipage}\tabularnewline
\bottomrule
\end{longtable}
